\section{Các phương pháp đánh giá mô hình}
\subsection{Ma trận nhầm lẫn (Confusion Matrix)}
$\indent$Ma trận nhầm lẫn là một bảng đơn giản được sử dụng để đo lường hiệu suất của mô hình phân loại. Nó so sánh các dự đoán do mô hình đưa ra với kết quả thực tế và cho thấy mô hình đã đúng hoặc sai ở đâu. Điều này giúp hiểu được nơi mô hình đang mắc lỗi để có thể cải thiện nó.

\subsubsection*{a) Các thành phần của Confusion Matrix ~\cite{confusionmatrix} }
Ma trận nhầm lẫn phân tích các dự đoán thành bốn danh mục chính:

\begin{itemize}
    \item \textbf{Dương tính Thực (True Positive - TP)}: Mô hình dự đoán đúng kết quả dương tính, tức kết quả thực tế là dương tính.
    \item \textbf{Âm tính Thực (True Negative - TN)}: Mô hình dự đoán đúng kết quả âm tính, tức kết quả thực tế là âm tính.
    \item \textbf{Dương tính Giả (False Positive - FP)}: Mô hình dự đoán sai kết quả dương tính, tức kết quả thực tế là âm tính. Còn được gọi là \textbf{Lỗi Loại I (Type I Error)}.
    \item \textbf{Âm tính Giả (False Negative - FN)}: Mô hình dự đoán sai kết quả âm tính, tức kết quả thực tế là dương tính. Còn được gọi là \textbf{Lỗi Loại II (Type II Error)}.
\end{itemize}


\begin{figure}[H]
    \centering
    \includegraphics[width=0.6\textwidth]{Images/pp.png}
    \vspace{15pt}
    \caption{Các chỉ số đánh giá}
    \label{fig:basic_cm}
\end{figure}

\subsubsection*{b) Biểu diễn Confusion Matrix}
\begin{table}[H]
\centering
\renewcommand{\arraystretch}{1.3}
\setlength{\tabcolsep}{10pt}
\begin{tabular}{|c|c|c|c|}
\hline
\multirow{2}{*}{\textbf{Thực tế (Actual)}} & \multicolumn{2}{c|}{\textbf{Dự đoán (Predicted)}} & \multirow{2}{*}{\textbf{Tổng}} \\ \cline{2-3}
 & \textbf{Positive} & \textbf{Negative} & \\ \hline
\textbf{Positive} & TP (True Positive) & FN (False Negative) & TP + FN \\ \hline
\textbf{Negative} & FP (False Positive) & TN (True Negative) & FP + TN \\ \hline
\textbf{Tổng} & TP + FP & FN + TN & TP + TN + FP + FN \\ \hline
\end{tabular}
\caption{Confusion Matrix cho bài toán phân loại nhị phân}
\label{table:confusion_matrix}
\end{table}

\subsubsection*{c) Các chỉ số đánh giá dựa trên Confusion Matrix}
$\indent$\textbf{Accuracy (Độ chính xác)}
Độ chính xác cho thấy có bao nhiêu dự đoán mà mô hình đã đúng trong tổng số các dự đoán. Nó cho ý tưởng về hiệu suất tổng thể nhưng có thể gây hiểu nhầm khi một lớp chiếm ưu thế hơn lớp khác.

\[
Accuracy = \frac{TP + TN}{TP + TN + FP + FN}
\]

\textbf{Precision (Độ chính xác dương)}
Precision tập trung vào chất lượng của các dự đoán dương tính của mô hình. Nó cho chúng ta biết có bao nhiêu dự đoán "dương tính" thực sự chính xác.

\[
Precision = \frac{TP}{TP + FP}
\]

\textbf{Recall (Độ nhạy)}
Recall đo lường mô hình tốt như thế nào trong việc dự đoán các trường hợp dương tính. Nó cho thấy tỷ lệ các trường hợp dương tính thực tế được phát hiện.

\[
Recall = \frac{TP}{TP + FN}
\]

\textbf{F1-score}
F1-score kết hợp Precision và Recall thành một chỉ số duy nhất để cân bằng sự đánh đổi giữa chúng. Nó cung cấp ý nghĩa tốt hơn về hiệu suất tổng thể của mô hình, đặc biệt đối với các tập dữ liệu mất cân bằng.

\[
F1 = \frac{2 \times Precision \times Recall}{Precision + Recall}
\]

\textbf{Specificity (Độ đặc hiệu)}
Specificity đo lường khả năng của mô hình trong việc xác định chính xác các trường hợp âm tính. Specificity còn được gọi là Tỷ lệ Âm tính Thực (True Negative Rate).

\[
Specificity = \frac{TN}{TN + FP}
\]

\textbf{Lỗi Loại I và Loại II (Type 1 and Type 2 Error)}
\begin{itemize}
    \item \textbf{Lỗi Loại I (Type 1 Error)}: Xảy ra khi mô hình dự đoán sai một trường hợp dương tính nhưng trường hợp thực tế là âm tính. Đây còn được gọi là \textbf{dương tính giả (false positive)}. Lỗi Loại I ảnh hưởng đến \textbf{precision} của mô hình.
    \[
    \text{Type 1 Error} = \frac{FP}{FP + TN}
    \]
    
    \item \textbf{Lỗi Loại II (Type 2 Error)}: Xảy ra khi mô hình không dự đoán được một trường hợp dương tính mặc dù nó thực sự là dương tính. Đây còn được gọi là \textbf{âm tính giả (false negative)}. Lỗi Loại II ảnh hưởng đến \textbf{recall} của mô hình.
    \[
    \text{Type 2 Error} = \frac{FN}{TP + FN}
    \]
\end{itemize}

\subsubsection*{d) Ví dụ minh họa}
Xét bài toán phân loại hình ảnh "Chó" và "Không phải chó" với 10 mẫu dữ liệu:

\begin{table}[H]
\centering
\begin{tabular}{|c|c|c|c|c|c|c|c|c|c|c|}
\hline
\textbf{Mẫu} & 1 & 2 & 3 & 4 & 5 & 6 & 7 & 8 & 9 & 10 \\ \hline
\textbf{Thực tế} & Chó & Chó & Chó & Không & Chó & Không & Chó & Chó & Không & Không \\ \hline
\textbf{Dự đoán} & Chó & Không & Chó & Không & Chó & Chó & Chó & Chó & Không & Không \\ \hline
\textbf{Kết quả} & TP & FN & TP & TN & TP & FP & TP & TP & TN & TN \\ \hline
\end{tabular}
\caption{Ví dụ chi tiết các dự đoán}
\end{table}
\begin{figure}[H]
    \centering
    \includegraphics[width=0.5\linewidth]{Images//Confusion Matrix/CMvd1.png}
    \vspace{15pt}
    \caption{Confusion Matrix cho bài toán phân loại}
    \label{fig:placeholder}
\end{figure}
\begin{table}[H]
\centering
\begin{tabular}{|c|c|c|c|}
\hline
\multirow{2}{*}{\textbf{Thực tế}} & \multicolumn{2}{c|}{\textbf{Dự đoán}} & \multirow{2}{*}{\textbf{Tổng}} \\ \cline{2-3}
 & \textbf{Chó} & \textbf{Không phải} & \\ \hline
\textbf{Chó} & TP = 5 & FN = 1 & 6 \\ \hline
\textbf{Không phải} & FP = 1 & TN = 3 & 4 \\ \hline
\textbf{Tổng} & 6 & 4 & 10 \\ \hline
\end{tabular}
\caption{Confusion Matrix tổng hợp}
\end{table}

Tính các chỉ số:
\begin{itemize}
    \item \textbf{Accuracy} = $\frac{5 + 3}{10} = 0.8$ (80\%)
    \item \textbf{Precision} = $\frac{5}{5 + 1} = 0.833$ (83.3\%)
    \item \textbf{Recall} = $\frac{5}{5 + 1} = 0.833$ (83.3\%)
    \item \textbf{F1-score} = $\frac{2 \times 0.833 \times 0.833}{0.833 + 0.833} = 0.833$ (83.3\%)
    \item \textbf{Specificity} = $\frac{3}{3 + 1} = 0.75$ (75\%)
\end{itemize}

\subsubsection*{e) Confusion Matrix cho bài toán Đa lớp (Multi-class)}
Trong phân loại đa lớp, ma trận nhầm lẫn được mở rộng để xem xét nhiều lớp:
\begin{itemize}
    \item \textbf{Hàng} đại diện cho các lớp thực tế (ground truth)
    \item \textbf{Cột} đại diện cho các lớp dự đoán
    \item Mỗi ô trong ma trận cho thấy tần suất một lớp thực tế cụ thể được dự đoán là một lớp khác
\end{itemize}


\begin{figure}[H]
    \centering
    \includegraphics[width=0.5\linewidth]{Images//Confusion Matrix/CMvd2.png}
    \vspace{15pt}
    \caption{Confusion Matrix cho bài toán phân loại 3 lớp}
    \label{fig:placeholder}
\end{figure}

\paragraph{Ví dụ: Phân loại hình ảnh Mèo, Chó, Ngựa}
\begin{table}[H]
\centering
\begin{tabular}{|c|c|c|c|c|}
\hline
\multirow{2}{*}{\textbf{Thực tế}} & \multicolumn{3}{c|}{\textbf{Dự đoán}} & \multirow{2}{*}{\textbf{Tổng}} \\ \cline{2-4}
 & \textbf{Mèo} & \textbf{Chó} & \textbf{Ngựa} & \\ \hline
\textbf{Mèo} & 8 & 1 & 1 & 10 \\ \hline
\textbf{Chó} & 2 & 10 & 0 & 12 \\ \hline
\textbf{Ngựa} & 0 & 2 & 8 & 10 \\ \hline
\textbf{Tổng} & 10 & 13 & 9 & 32 \\ \hline
\end{tabular}
\caption{Confusion Matrix cho bài toán phân loại 3 lớp}
\end{table}

Trong ví dụ này:
\begin{itemize}
    \item \textbf{Mèo}: 8 được xác định chính xác, 1 bị xác định nhầm thành chó, 1 bị xác định nhầm thành ngựa
    \item \textbf{Chó}: 10 được xác định chính xác, 2 bị xác định nhầm thành mèo
    \item \textbf{Ngựa}: 8 được xác định chính xác, 2 bị xác định nhầm thành chó
\end{itemize}
\subsection{Đường cong ROC và AUC~\cite{aucroc} }
\subsubsection*{a) Giới thiệu về đường cong ROC}
$\indent$Đường cong ROC (Receiver Operating Characteristic) là một công cụ trực quan được sử dụng để đánh giá hiệu suất của các mô hình phân loại nhị phân. Nó minh họa sự đánh đổi giữa hai tỷ lệ:
\begin{itemize}
    \item \textbf{Tỷ lệ Dương tính Thực (True Positive Rate - TPR)}: Còn được gọi là \textbf{Độ nhạy (Sensitivity)} hoặc \textbf{Recall}.
    \item \textbf{Tỷ lệ Dương tính Giả (False Positive Rate - FPR)}: Tỷ lệ các trường hợp âm tính bị phân loại sai.
\end{itemize}

\subsubsection*{b) Công thức tính toán}
Các chỉ số được sử dụng trong đường cong ROC được tính dựa trên Ma trận Nhầm lẫn (Confusion Matrix):

\[
TPR = \frac{TP}{TP + FN}
\]
\[
FPR = \frac{FP}{FP + TN}
\]
\[
Specificity = \frac{TN}{TN + FP} = 1 - FPR
\]

\subsubsection*{c) Ý nghĩa của đường cong ROC}
Đường cong ROC được tạo ra bằng cách vẽ TPR theo FPR tại \textbf{tất cả các ngưỡng phân loại có thể}. Mỗi điểm trên đường cong đại diện cho một ngưỡng quyết định khác nhau.
\begin{itemize}
    \item Điểm ở góc trên bên trái \((0, 1)\) biểu thị một bộ phân loại hoàn hảo (TPR=1, FPR=0).
    \item Đường chéo từ \((0,0)\) đến \((1,1)\) biểu thị hiệu suất của một bộ phân loại ngẫu nhiên, không có kỹ năng phân biệt.
    \item Một đường cong càng "cong" lên phía trên góc trên bên trái thì mô hình càng tốt.
\end{itemize}

\begin{figure}[H]
    \centering
    \includegraphics[width=0.5\linewidth]{Images//Confusion Matrix/rocauc.png}
    \vspace{15pt}
    \caption{Chỉ số đánh giá phân loại ROC-AUC}
    \label{fig:placeholder}
\end{figure}

\subsubsection*{d) AUC (Area Under the Curve)}
AUC là diện tích nằm dưới đường cong ROC, cung cấp một thước đo tổng hợp về khả năng phân loại của mô hình.
\[
AUC = \int_{0}^{1} TPR(FPR)  d(FPR)
\]

\subsubsection*{e) Diễn giải giá trị AUC}
\begin{itemize}
    \item \textbf{AUC = 1.0}: Mô hình hoàn hảo, có khả năng phân biệt tuyệt đối giữa hai lớp.
    \item \textbf{AUC = 0.9 - 0.99}: Mô hình xuất sắc.
    \item \textbf{AUC = 0.8 - 0.89}: Mô hình tốt.
    \item \textbf{AUC = 0.7 - 0.79}: Mô hình trung bình.
    \item \textbf{AUC = 0.5}: Mô hình không có khả năng phân biệt, tương đương với đoán ngẫu nhiên.
    \item \textbf{AUC < 0.5}: Mô hình hoạt động tệ hơn cả đoán ngẫu nhiên; có thể đảo ngược dự đoán để cải thiện.
\end{itemize}

\subsubsection*{f) Khi nào sử dụng AUC-ROC?}
AUC-ROC đặc biệt hữu ích khi:
\begin{itemize}
    \item Tập dữ liệu có sự cân bằng tương đối giữa các lớp.
    \item Chi phí của False Positive (FP) và False Negative (FN) được coi là tương đương.
    \item Khi muốn đánh giá hiệu suất mô hình trên nhiều ngưỡng phân loại khác nhau mà không cần chọn một ngưỡng cụ thể.
\end{itemize}

\textbf{Lưu ý}: Đối với các tập dữ liệu mất cân bằng nghiêm trọng, đường cong Precision-Recall thường cung cấp đánh giá thực tế hơn so với AUC-ROC.

\subsection{Chỉ số KS (Kolmogorov--Smirnov Statistic)}
$\indent$Chỉ số Kolmogorov--Smirnov (KS) là một thống kê phi tham số dùng để so sánh hai phân phối xác suất một chiều. Trong thống kê cổ điển, KS được sử dụng để kiểm tra xem một mẫu có tuân theo một phân phối lý thuyết cho trước (one–sample KS test) hoặc hai mẫu độc lập có cùng phân phối hay không (two–sample KS test). Trong bối cảnh học máy và chấm điểm tín dụng, ta thường quan tâm trực tiếp tới \textbf{giá trị thống kê KS} như một thước đo cho khả năng tách biệt giữa hai lớp (ví dụ: khách hàng tốt và khách hàng xấu).

\subsubsection*{a) Hàm phân phối tích luỹ thực nghiệm (Empirical Distribution Function)}
Giả sử ta có một mẫu quan sát $X_1, X_2, \dots, X_n$. Hàm phân phối tích luỹ thực nghiệm (EDF) của mẫu tại một giá trị $x$ được định nghĩa là:
\[
F_n(x) = \frac{1}{n} \sum_{i=1}^{n} \mathbf{1}_{\{X_i \le x\}},
\]
trong đó $\mathbf{1}_{\{X_i \le x\}} = 1$ nếu $X_i \le x$ và bằng $0$ nếu ngược lại. $F_n(x)$ biểu diễn tỷ lệ các điểm dữ liệu có giá trị không vượt quá $x$ trong mẫu.

\subsubsection*{b) Định nghĩa thống kê KS}
\begin{itemize}
    \item \textbf{One–sample KS}: dùng để kiểm tra xem mẫu có tuân theo một phân phối lý thuyết $F(x)$ cho trước hay không. Thống kê KS được định nghĩa là:
    \[
    D_n = \sup_{x} \left| F_n(x) - F(x) \right|.
    \]
    \item \textbf{Two–sample KS}: dùng để so sánh hai mẫu độc lập có kích thước lần lượt là $n_1$ và $n_2$, với các hàm phân phối thực nghiệm $F_{1,n}(x)$ và $F_{2,m}(x)$. Khi đó:
    \[
    D_{n_1,n_2} = \sup_{x} \left| F_{1,n}(x) - F_{2,m}(x) \right|.
    \]
\end{itemize}
Giá trị $D$ càng lớn thì khoảng cách giữa hai phân phối càng lớn, nghĩa là hình dạng phân phối khác nhau nhiều hơn.

\subsubsection*{c) Liên hệ với bài toán phân loại nhị phân}
Trong bài toán chấm điểm tín dụng hoặc phân loại nhị phân, mô hình thường trả về một \textbf{điểm số} (score) hoặc \textbf{xác suất} cho mỗi quan sát. Khi đó, ta có thể:
\begin{itemize}
    \item Tách các điểm số thành hai nhóm: nhóm lớp dương tính (ví dụ: khách hàng xấu) và nhóm lớp âm tính (khách hàng tốt).
    \item Xây dựng hai hàm phân phối tích luỹ thực nghiệm $F_{\text{pos}}(s)$ và $F_{\text{neg}}(s)$ cho điểm số $s$.
\end{itemize}
Chỉ số KS trong ngữ cảnh này chính là \textbf{độ lệch tối đa giữa hai hàm phân phối tích luỹ}:
\[
KS = \sup_{s} \left| F_{\text{pos}}(s) - F_{\text{neg}}(s) \right|.
\]

Mặt khác, từ quan điểm đường cong ROC, tại mỗi ngưỡng $t$ ta có:
\[
TPR(t) = \frac{TP(t)}{TP(t) + FN(t)}, \quad FPR(t) = \frac{FP(t)}{FP(t) + TN(t)}.
\]
Có thể chứng minh rằng (với cách sắp xếp ngưỡng phù hợp), thống kê KS tương đương với:
\[
KS = \max_{t} \left| TPR(t) - FPR(t) \right|.
\]
Do đó, KS có thể xem như \textbf{khoảng cách lớn nhất} giữa hai đường CDF của lớp dương và lớp âm, hoặc giữa hai đường TPR và FPR khi thay đổi ngưỡng phân loại.

\subsubsection*{d) Diễn giải giá trị KS trong đánh giá mô hình}
Trong thực tế ứng dụng (đặc biệt là chấm điểm tín dụng), KS thường được diễn giải theo trực giác:
\begin{itemize}
    \item $KS$ càng lớn $\Rightarrow$ hai phân phối điểm của hai lớp càng tách biệt rõ rệt $\Rightarrow$ mô hình càng dễ phân biệt khách hàng tốt và xấu.
    \item Một số ngưỡng tham khảo thường dùng:
    \begin{itemize}
        \item $KS < 0.2$: mô hình yếu, khả năng tách biệt kém.
        \item $0.2 \le KS < 0.4$: mô hình ở mức chấp nhận được.
        \item $KS \ge 0.4$: mô hình tốt, phân biệt hai lớp rõ.
    \end{itemize}
\end{itemize}
Các ngưỡng này chỉ mang tính kinh nghiệm, cần được hiệu chỉnh tuỳ theo bài toán và yêu cầu của doanh nghiệp.

\subsubsection*{e) Ưu điểm và hạn chế}
\textbf{Ưu điểm:}
\begin{itemize}
    \item Là một kiểm định phi tham số, không yêu cầu giả định về dạng phân phối (không cần giả định chuẩn, mũ, \dots).
    \item Chỉ số đơn giản, trực quan, có thể nhìn trực tiếp trên biểu đồ hai CDF hoặc từ đường cong ROC.
    \item Rất phổ biến trong các hệ thống chấm điểm tín dụng để so sánh khả năng phân biệt giữa các mô hình.
\end{itemize}

\textbf{Hạn chế:}
\begin{itemize}
    \item Nhạy với kích thước mẫu: với mẫu rất lớn, các khác biệt nhỏ cũng có thể cho giá trị KS cao; với mẫu nhỏ, kiểm định mất sức mạnh.
    \item Thiết kế chủ yếu cho dữ liệu liên tục; khi áp dụng cho dữ liệu rời rạc cần xử lý cẩn thận.
    \item Chỉ phản ánh \textbf{khoảng cách tối đa} giữa hai phân phối, không cho biết chi tiết toàn bộ hình dạng phân phối như thế nào.
\end{itemize}
